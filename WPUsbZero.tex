\documentclass[a4paper,11pt,DIV=11,BROC=5mm,bigheadings,idxtotoc,cleardoubleempty,halfparskip,oneside,openright]{scrreprt} % 

\usepackage{datetime}

\usepackage{graphicx}
\usepackage{xcolor}

\usepackage{blindtext}

\usepackage{cite}% Zitieren
\usepackage{bibgerm}% Literatur in Deutscher DIN
\usepackage{url}


\usepackage{amsmath}
\usepackage{amssymb}

\usepackage{listings}

\usepackage[utf8]{inputenc}
\usepackage{lmodern}

\usepackage{microtype}


\usepackage{scrpage2} 	% Kopf & Fußzeile im KOMA Stil
\pagestyle{scrheadings}	% Aktiviert Verwendung vordefinierter Kolumnentitel
\clearscrheadfoot 		% alle Standard-Werte und Formatierungen löschen
\setkomafont{pagehead}{\scshape}	% Schriftart in Kopfzeile, \scshape = Kapitelchen
\automark[chapter]{section} % [linke Seite]{rechte Seite}
%\ohead{\def\pagestyle{PDTS}{\hrulefill\includegraphics[width = 6cm]{bilder/thi_logo_quer_cropped}}}
\ohead{\includegraphics[width = 6cm]{bilder/thi_logo_quer_cropped}}

%\ihead{\textsc{Abschlussarbeit}}
\ihead{\headmark}

%\setheadwidth[0pt]{textwithmarginpar}
\ofoot{\vspace{-0.3cm} \pagemark} 						
\ifoot{\vspace{-0.3cm} Dominik Gunther Florian Schlecht} 
				
%\setheadtopline{2pt}				
%\setheadsepline{.4pt}
\setfootsepline{.4pt}	% Trennlinie Fußzeile und Textkörper

%-------------------------------------------------------------






%------------------------------------------------------------------
%% Längenanpassungen
%------------------------------------------------------------------

\setlength{\headsep}{10mm}				% Textabstand zur Kopfzeile
\setlength{\footskip}{15mm}				% Abstand zur Fußzeile
\setlength{\parindent}{0em}				% Einzug nach Absatz

%------------------------------------------------------------------
%% THI - Farbdefinitionen
%------------------------------------------------------------------

% Farben der THI
\definecolor{THIblue}{rgb}{0.0078,0.1176,0.4705}
%%-------------Allgemeine Definitionen----------------------------------
% Farbige Aufwertung der berschriften
\addtokomafont{chapter}{\color{THIblue}}
\addtokomafont{section}{\color{THIblue}}
\addtokomafont{caption}{\color{THIblue}}
\addtokomafont{subsection}{\color{THIblue}}
\addtokomafont{subsubsection}{\color{THIblue}}
\setkomafont{captionlabel}{\color{THIblue}}

%------------------------------------------------------------------
%% Interessante optinale Packages
%------------------------------------------------------------------

% -> Verlinkter Text im PDF
	\usepackage[
			pdftex,
			linkcolor=THIblue,			% Farbe der Verlinkung
			linktocpage=true,			% Im TOC wird Seitenzahl verlinkt(true),bzw. Text(false)
			colorlinks=true,			% 'true' keinen Kasten um Link
	%		citecolor=blue,
	%		pdfhighlight=/P,
	%		bookmarks,
	%		hyperfigures=true,
	%		citebordercolor={0 0 1},
	%		linkbordercolor={0 0 1},
	%		menubordercolor={0 0 1},
	%		backref=true,
	%		pagebackref=true,
	%%		bookmarksopen,
	%		bookmarksnumbered,
	%		pdfpagelabels=false,
	%		pdfstartpage=1,
	%		pdfstartview=Fit,			% Modus beim Öffnen (Fit = An Seitengröße anpassen)
			pdftitle={Umgehen von PID und VID basierten USB-Policies},
			pdfauthor={Dominik Schlecht},
	%		pdfstartview=Fit,
	%		pdfdisplaydoctitle=true,
	%		plainpages=false
				]{hyperref}  



%------------------------------------------------------------------
%% Wichtige Definition für Aufteilung von Formelverzeichnis und Abkürzungsverzeichnis
%------------------------------------------------------------------

% Nomenklaturverzeichnis, Formelzeichenliste Anpassungen für nomenclature: damit lassen sich zwei getrennte Symbolverzeichnisse anlegen, ziehmlich cool!
\usepackage[intoc,compatible,german]{nomencl}	
		\renewcommand{\nomgroup}[1]{	\ifthenelse{\equal{#1}{A}}{\item[{\normalfont\sffamily\bfseries\LARGE\textcolor[rgb]{0,0.112,0.47}{Abkürzungen{\phantom{\Huge $\frac{\frac{\frac{A}{a}}{a}}{\frac{a}{a}}$}}}}]}{	\ifthenelse{\equal{#1}{A}}{\item[{\normalfont\sffamily\bfseries\LARGE\textcolor[rgb]{0,0.112,0.47}{Formelzeichen{\phantom{\Huge $\frac{A}{\frac{a}{a}}$}}}}]}{}}}
		
%% z.B. (1)  \nomenclature{$v_0$}{Startgeschwindigkeit}   -> Formelverzeichnis
\nomenclature[A]{SAF}{Security Awareness Framework}



%------------------------------------------------------------------
%% Anpassung von Abständen, Längen vom Nomenclaturverzeichnis (Abkürzungs- und Formelverzeichnis)
%------------------------------------------------------------------

		% Abstand zwischen Einträgen im Symbolverzeichnis (-\parsep = 0)
		\setlength{\nomitemsep}{-\parsep} 
		\setlength{\nomlabelwidth}{5em}	
		\renewcommand{\nomlabel}[1]{#1 \dotfill}  % Punkte im zwischen Nummer und Kapiteleintrag






%------------------------------------------------------------------
%% Definition Standardordner der Bilddatein für schnellen Zugriff
%------------------------------------------------------------------
%
%\newcommand\figpfad{bilder/}
%


%%-------------Silbentrennung--------------------------------------
\hyphenation{}

%%-------------Index-----------------------------------------------
\makeindex
\makenomenclature

%------------------------------------------------------------------
% Angabe der zu verwendenden Dateien, sodass nur z.B. das aktuelle 
% Dokument compiliert wird; den Rest mit % auskommentieren
		\includeonly{%
			chapter/title_page,%
			chapter/thanks_statement,
            chapter/einleitung,
			chapter/appendix%
		}


%------------------------------------------------------------------------

%-----------------------------------------------------------------
%---------------Dokumentenbeginn----------------------------------
%-----------------------------------------------------------------
	\begin{document}
			
			
% Titelseite
\begin{titlepage}

\phantom{tmpText}

\vspace{1cm}

\begin{figure}[h!]
\centering
\includegraphics[width=\textwidth]{bilder/thi_logo_cropped.pdf}
\end{figure}

  \begin{center}

%\vspace{1cm}
    
    
    \textbf{{\large Seminararbeit/Whitepaper} \\[3ex]
    {\LARGE Umgehen von PID und VID basierten USB-Policies} \\[1ex]
    %
    \vfill
    %
    angefertigt von \\
    Dominik Schlecht \\[2ex] %Vorname Nachname
    %
    \vfill
    %
    Betreuer:} \\%[5ex]
    \begin{tabular}{ll}
      Technische Hochschule Ingolstadt: & Dr. Stanislaus \\
      Allianz Deutschland AG: & Dr. Stremmel und Herr Gerhager
    \end{tabular} \\[2ex]
    %
    \vfill
    %
    Ingolstadt, \today
  \end{center}
\end{titlepage}
			
				\cleardoublepage

			
%------------------------------------------------------------------------

% Nummerierung
\pagenumbering{roman}

% Sperrvermerk
\chapter*{Sperrvermerk}
Die vorliegende Arbeit \glqq Umgehen von USB-Deskriptor basierten USB-Policies am Beispiel einer virtuellen Umgebung\grqq\ wurde mit Unterstützung der Allianz Deutschland AG erstellt. Es sind jedoch keine internen, vertraulichen oder streng vertraulichen Informationen enthalten. Die Weitergabe des Inhalts der Arbeit im Gesamten oder in Teilen sowie das Anfertigen von Kopien oder Abschriften -- auch in digitaler Form -- sind grunds\"{a}tzlich erlaubt.
\addcontentsline{toc}{chapter}{Sperrvermerk}

% Erkl\"{a}rung
\chapter*{Erkl\"{a}rung}
Hiermit erkl\"{a}re ich, dass ich die vorliegende Seminararbeit bis auf die offizielle Betreuung durch den Aufgabensteller selbstst\"{a}ndig und ohne fremde Hilfe verfasst habe.\par
Die verwendeten Quellen sowie die verwendeten Hilfsmittel sind vollst\"{a}ndig angegeben. W\"{o}rtlich \"{u}bernommene Textteile und \"{u}bernommene Bilder und Zeichnungen sind in jedem Einzelfall kenntlich gemacht. \\[10ex]
Ingolstadt, \today
\addcontentsline{toc}{chapter}{Erkl\"{a}rung}

% Danksagung (optional)
\chapter*{Danksagung}
An dieser Stelle möchte ich mich bei allen Bedanken, die mich bei der Erstellung dieser Arbeit unterstützt haben. Besonderer Dank dabei gebührt den Kollegen aus der Allianz Deutschland AG und meinen Eltern.

\begin{flushright}
\sffamily Dominik Schlecht, \today
\end{flushright}
\addcontentsline{toc}{chapter}{Danksagung}


%------------------------------------------------------------------------

				\cleardoublepage	
					
			% Inhaltsverzeichnis
			%\setcounter{\secnumdepth}{2} % Zwei Ebenen im Inhaltsverzeichnis auflisten
			\tableofcontents
				\cleardoublepage	
		
			% Abkrzungsverzeichnis
			\printnomenclature
				\cleardoublepage	
			
%------------------------------------------------------------------------
			% Neunummerierung des Hauptteils
			\pagenumbering{arabic}
			\setcounter{page}{1}            
            \cleardoublepage
            
            \chapter{Einleitung}
In Zeiten von Heartbleed und Shellshock, Snowden und der NSA und der fortlaufenden Digitalisierung der Industrie und Gesellschaft wird das Thema Informationssicherheit immer wichtiger. Daten werden, unabhängig davon, ob diese Privatpersonen oder Firmen zugeordnet sind, immer wertvoller. So ergeben sich Beispielsweise aus einem gehackten Smartphone einer Privatperson Informationen wie E-Mail-Adressen, Kontakten und Chatverläufen bis hin zu Passwörter für das Online-Banking oder persönlichen Bildern. Da diese Informationen auf dem Schwarzmarkt oft zu nicht zu vernachlässigenden Preisen gehandelt werden, hat sich hier ein neuer Markt entwickelt. Black-Hat-Hacker oder Gruppen versuchen über Phishing-Mails oder Virenkampagnen so viele Daten wie Möglich zu sammeln, um diese im Anschluss zu verkaufen. Diese Tätigkeiten werden oft unter dem Schlagwort Cybercrime zusammengefasst. Umso kritischer in Hinsicht auf den möglichen finanziellen Schaden ist Cybercrime jedoch, wenn es um Unternehmen geht. Würde als Beispiel die Webseite eines Versandunternehmens von Hackern offline genommen werden, entstehen reale Verluste, da die Käufer auf andere Shops wechseln. Ähnlich kritisch wäre es, wenn streng vertrauliche Dokumente von Unternehmen, wie z.B. Konstruktionsskizzen oder Quellcode, durch Hacker erbeutet und an ein Konkurrenzunternehmen verkauft würden. So schätzt die Firma McAffee im Jahr 2014 den Verlust für die Wirtschaft durch "Cybercrime" auf bis zu 575 Milliarden USD\footnote{Siehe: $http://csis.org/files/attachments/140609\_McAfee\_PDF.pdf$}. 
Um diesem Trend entgegen zu wirken, müssen Unternehmen Maßnahmen ergreifen, welche das Schutzniveau erhöhen. Oft wird hier nur auf die technische Seite geachtet, z.B. auf das schnelle Patching von Servern. Dies ist sicherlich ein essentielle Bestandteil, oft wird jedoch die menschliche Komponente vernachlässigt, obwohl Social Engineering wohl zu den häufigsten Angriff Szenarien zählt. Dabei ist dies oft der erste Einfallspunkt für Cyberkriminelle. Für professionelle Hacker ist es kein Problem, eine E-Mail mit gefälschtem Absender und einem noch nicht erkannten Virus im Anhang zu verschicken. Öffnet der Mitarbeiter der Firma, welcher eine solche E-Mail empfängt, den Anhang ergeben sich gute Chancen für den Hacker Zugriff in das interne Netzwerk der Firma zu erlangen. Um dieses Risiko zu minimieren werden bei der Allianz Deutschland AG, im weiteren Verlauf AZD genannt, Awareness-Kampagnen mit Postern und Hinweistafeln, jedoch auch Schulungen mit Themen wie Spam und Phishing, organisiert. Diese Vorträge werden von wechselnden Mitarbeitern der Sicherheitsabteilung gehalten, welche im vorhinein Folien mit PowerPoint entwerfen und Demonstration entwickeln, an welchen das Themengebiet den Zuhörern praxisnah erläutern werden kann. Findet dieses ohne ein zentrales Programm statt, ergeben sich Problemfelder wie der Austausch, die Aktualität oder der Qualität der Vortragsmaterialien. Bei diesen und anderen Problemen soll das Security Awareness Framework, kurz SAF, Abhilfe schaffen.
\\
\\
\\
$[TODO ->]$Diese sind dann oft veraltet oder instabil. Zudem werden die Demos oft unter den Kollegen der Sicherheitsabteilungen nicht ausgetauscht. Dies hat mehrere Gründe, angefangen davon, dass diese oft nur für den Ersteller verständlich sind bis hin zu technischen Problemen wie der Größe oder dem Format der Virtuellen Maschine, auf welchen die Demos oft gespeichert sind. Diese Probleme soll durch das Security Awareness Framework gelöst werden.
                \cleardoublepage
                
			\chapter{Motivation}
Bei einer Prüfung interner Regularien bei der Informationssicherheit wurde das Thema USB-Geräte in Verbindung mit Thin oder auch sogenannten Zero-Clients aufgegriffen. Hier soll aus gegeben fachlichen Anlässen eine Möglichkeit geschaffen werden lokale USB-Geräte, wie z.B. USB-Sticks oder USB-CD-Laufwerke, an die virtuelle Maschine des Benutzers durch zu stellen. Hier galt es das Risiko zu Prüfen und entsprechende Gegenmaßnahmen zu entwicklen. Würde man das Durchstellen von USB-Geräten jedoch im Allgemeinen erlauben, so würden sich erheblich Gefahren ergeben, welche unter \ref{AllgGefBeiUSB} erläutert werden. Um dem Vorzubeigen, soll auf Basis einer Policie, welche in \ref{Policies} weiter erläutert werden, das Durchstellen auf bestimmte Geräte begrenzt werden. Dies geschieht im normalfall über die Filterung nach PID und VID, welche eine Produkt, z.B. einen bestimmten USB-Stick, eindeutig Identifiezieren sollten. Diese Felder werden unter \ref{FelderUSB} weiter erläutert. Bei der durchgeführten Sicherheitsprüfung stellte sich jedoch heraus, dass diese Filterung durch die Fälschung der PID und VID leicht gefälscht werden können. Der Angriff, \ref{Angriff}, soll in diesem Dokument erläuter und Aufgezeigt werden. Den Proof-Of-Concept finden Sie untern \ref{PoC}.
		
			\chapter{Allgemeine Gefahren bei USB}\label{AllgGefBeiUSB}
USB-Geräte stellen Gefahren auf verschiedenen Ebenen dar. Zum einen werden USB-Sticks, zumindest in dem Szenario, das hier betrachtet wurde, von Dritten an Mitarbeiter gegeben. Das heist, dass der Dritte, insofern dieser die nötige kriminielle Energie aufweist, ein prepariertes Gerät einschicken könnte. Erschwerend kommt hinzu, dass der Mitarbeiter keine Möglichkeit hat, ein böses USB-Gerät von einem normalen zu unterscheiden. Auch besteht die Möglichkeit für den Dritten den Angriff über längerere Zeit vorzubereiten, da kein Zeitdruck besteht, im Gegenteil, USB-Geräte sind gerade noch im Aufwärtstrend. TODO CD vs USB Statistik. Bei der Manipulation sind verschiedene Szenarien denkbar. Einige sind in den nächsten Sektionen aufgeführt.

			\section{Viren}
Viren sind eine wachsende Bedrohung in der heutigen Zeit. Vor einigen Jahren waren einige wenige Virenfamilien weit verbreitet. So konnten Virenhersteller über signaturbasierte Suchalgorithmen nach bekannten Mustern suchen und Viren identifizieren. In den letzten Jahren zeichnet sich jedoch der Trend ab, dass Viren sich schneller weiterentwickeln und zudem oft polymorph programmiert sind, also ihr aussehen bei einer Infektion verändern. Dadurch werden signaturbasierte Erkennungen immer uneffizienter und die Gefahr, dass ein Rechner unerkannt Infiziert wird, steigt. Eine Infektion passiert zumeist über sogenannte Browserexploits, also preparierte Webseiten, welche Lücken in der Software des Users nutzten, oder durch E-Mails verbreitet. In dem von uns betrachteten Szenario würde ein krimineller Dritter oder aber auch ein unwissinder Dritter, dessen Rechner von einem Virus infiziert ist, einen USB-Stick mit einem Virus einschicken. Dabei wird im folgenden zwischen Viren im ursprünglichen Sinn und Viren, welche sich auf der Treiberebene einnissten, unterschieden.

			\subsection{Normale Viren}
			\subsection{Viren auf Treiberebene und neue Möglichkeiten durch Virtualisierungsumgebungen}

			\section{Dateneinschleußung}

			
			\section{Datenabfluss}
Neben den Gefahren von außen müssen jedoch auch sogenannte \glqq Inside-Threaths\grqq beachtet werden. Dies wären Mitarbeiter, welche z.B. interne IT-Systeme manipulieren, um sich Vorteile oder Reichtümer zu verschaffen. Das wohl bekannteste Beispiel wäre hier wohl ein TODO Kevin Midnick Geschichte. Bezogen auf USB wäre ein Risiko der Abfluss von Daten, also wenn ein Mitarbeiter Daten auf einem USB-Stick speichert und diese mit nach Hause nimmt, um diese zu Verkaufen oder sich zu bereichern. Denkbar wären zum Beispiel Kundendaten, bei Admins Passwörter, Geschäftsberichte oder sonstige Unternehmensgeheimnisse. Diese können oft für Geld in einschlägigen Bereichen des Internets verkauft oder bei Geschäftsberichten zur Manipulation am Finanzmarkt genutzt werden. Auch wäre eine Abwerbung eines Mitarbeiters von einem anderen Unternehmen für Industriespionage denkbar.
Eine neue Bedrohung sind Geheimdienste, welche Personen in eine Unternehmen einschleußen oder Mitarbeiter abwerben, um Daten übe die Kunden zu sammeln. Dies wurde erst letztens durch von Edward Snowden veröffentlichte Dokumente publik. TODO Quelle

			\chapter{Policies} \label{Policies}
Policies sind Regelwerke, welche in IT-Systemen, aber auch als interne Regelungen in Unternehmen zum Einsatz kommen. TODO
						
			
			\chapter{Felder bei USB} \label{FelderUSB}
			\section{Allgemein}
			\section{PID}
			\section{VID}
			
			\chapter{Umgehung der Policies}\label{Angriff}
			\section{Teensy}
			\section{Vorgehen}
			\section{Proof of Concept} \label{PoC}

			\chapter{Ausblick in die Zukunft}			
			
			\chapter{Gegenmaßnahmen}
	
			
%------------------------------------------------------------------------
			
			% Anhang
			
%------------------------------------------------------------------------
% Anhänge
\appendix
\chapter{Appendix}
\section{Quellcode}

\lstset{% general command to set parameter(s)
	basicstyle=\small\ttfamily,%\small, % print whole listing small
	keywordstyle=\color{THIblue}\bfseries,% underlined boldblack keywords
	numbers=left,
	numberstyle=\color{gray},
	numbersep=5pt,
	captionpos=t}

\lstset{language=C}
\lstinputlisting[caption={Blink\_2.ino},language=c]{documents/Blink_2.ino}

\newpage
\section{Ergänzende Grafiken}

\begin{figure}[htbp]

\centering
\includegraphics[width=\textwidth]{bilder/EinstellungenArduino.png}
\label{fig:EinstellungenArduino}
\caption{Einstellungen Arduino}

\end{figure}

			
			% Literaturliste
			\newpage
			\bibliographystyle{plaindin}	% DIN Norm für Literaturdarstellung
			\bibliography{ref/ref_liste}	% Pfad und Datei der Ref-Datenbank
		
	\end{document}
%-----------------------------------------------------------------
