\chapter{Einleitung}
In Zeiten von \textit{Heartbleed}\footnote{http://cve.mitre.org/cgi-bin/cvename.cgi?name=CVE-2014-0160} und \textit{Shellshock}\footnote{http://www.fireeye.com/blog/technical/2014/09/Shellshock-in-the-wild.html}, \textit{Snowden} und der \textit{NSA}\footnote{http://www.theguardian.com/us-news/edward-snowden} und der fortlaufenden Digitalisierung der Industrie und Gesellschaft wird das Thema Informationssicherheit immer wichtiger. Daten werden, unabhängig davon, ob diese Privatpersonen oder Unternehmen zugeordnet sind, immer wertvoller. So ergeben sich beispielsweise aus einem gehackten Smartphone einer Privatperson Informationen wie E-Mail-Adressen, Kontakte und Chat-Verläufe bis hin zu Passwörter für Online-Banking oder persönlichen Bildern. Wenn diese Informationen auf dem Schwarzmarkt verkauft oder online veröffentlicht werden, kann dies für die Personen oft Reputation wie auch finanzielle Schäden nach sich ziehen. Diese Tätigkeiten werden oft unter dem Schlagwort Cybercrime zusammengefasst. Betrachtet man Unternehmen, so ist der möglichen finanziellen Schaden wesentlich höher als für Privatpersonen. Durch die Entwendung von Kreditkarten erlitten zum Beispiel mehrere Supermärkte in den USA beträchtliche Reputationsschäden TODO Quelle. Eventuell noch höhere Schäden könnte es nach sich ziehen, wenn streng vertrauliche Dokumente von Unternehmen, wie z.B. Konstruktionsskizzen für ein neues Automodell, Quellcode oder vorläufige Geschäftsberichte, durch Hacker erbeutet und an ein Konkurrenzunternehmen verkauft würden. Dies hört sich irreal an, aber die die Firma McAffee schätzt den Verlust für die Wirtschaft durch \glqq Cybercrime\grqq im Jahr 2014 auf bis zu 575 Milliarden USD\footnote{Siehe: $http://csis.org/files/attachments/140609\_McAfee\_PDF.pdf$}. 
Um diesem Trend entgegen zu wirken, müssen Unternehmen Maßnahmen ergreifen, welche das Schutzniveau erhöhen. Oft werden hier auf der technischen Seite nur internetseitige Komponenten beachtet, wie das schnelle Patching von Servern. Dies ist in Hinsicht auf  Poodle und Shellshock sicherlich auch notwendig, jedoch sollte man alle Wege, über welche Daten von Dritten in das Unternehmen gelangen, Daten an Dritte weitergeben werden könnten, sowie auch interne Bedrohungen wahrnehmen, einschätzen und eindämmen. Eine solche Prüfung war die Grundlage für dieses Dokument.