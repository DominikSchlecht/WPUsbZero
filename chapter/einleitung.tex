\chapter{Einleitung}
In Zeiten von Heartbleed und Shellshock, Snowden und der NSA und der fortlaufenden Digitalisierung der Industrie und Gesellschaft wird das Thema Informationssicherheit immer wichtiger. Daten werden, unabhängig davon, ob diese Privatpersonen oder Firmen zugeordnet sind, immer wertvoller. So ergeben sich Beispielsweise aus einem gehackten Smartphone einer Privatperson Informationen wie E-Mail-Adressen, Kontakten und Chatverläufen bis hin zu Passwörter für das Online-Banking oder persönlichen Bildern. Da diese Informationen auf dem Schwarzmarkt oft zu nicht zu vernachlässigenden Preisen gehandelt werden, hat sich hier ein neuer Markt entwickelt. Black-Hat-Hacker oder Kollektive versuchen über Phishing-Mails oder Virenkampagnen so viele Daten wie Möglich zu sammeln, um diese im Anschluss zu verkaufen oder diese Mittel für weitere Operationen zu nutzen. Diese Tätigkeiten werden oft unter dem Schlagwort Cybercrime zusammengefasst. Umso kritischer in Hinsicht auf den möglichen finanziellen Schaden ist Cybercrime jedoch, wenn es um Unternehmen geht. Durch die Entwendung von Kreditkarten erlitten zum Beispiel mehrere Supermärkte in den USA beträchtliche reputationsschäden TODO Quelle. Ähnlich kritisch wäre es, wenn streng vertrauliche Dokumente von Unternehmen, wie z.B. Konstruktionsskizzen oder Quellcode, durch Hacker erbeutet und an ein Konkurrenzunternehmen verkauft würden. So schätzt die Firma McAffee im Jahr 2014 den Verlust für die Wirtschaft durch "Cybercrime" auf bis zu 575 Milliarden USD\footnote{Siehe: $http://csis.org/files/attachments/140609\_McAfee\_PDF.pdf$}. 
Um diesem Trend entgegen zu wirken, müssen Unternehmen Maßnahmen ergreifen, welche das Schutzniveau erhöhen. Oft wird hier auf der technischen Seite nur internetseitige Komponenten geachtet, wie das schnelle Patching von Servern. Dies ist sicherlich ein essentieller Bestandteil, jedoch sollte man alle Wege, über welche Daten von Dritten in das Unternehmen gelagen, sowie auch interne Bedrohungen wahrnehmen und eindämmen.