\chapter{Einleitung}
In Zeiten von Heartbleed und Shellshock, Snowden und der NSA und der fortlaufenden Digitalisierung der Industrie und Gesellschaft wird das Thema Informationssicherheit immer wichtiger. Daten werden, unabhängig davon, ob diese Privatpersonen oder Firmen zugeordnet sind, immer wertvoller. So ergeben sich Beispielsweise aus einem gehackten Smartphone einer Privatperson Informationen wie E-Mail-Adressen, Kontakten und Chatverläufen bis hin zu Passwörter für das Online-Banking oder persönlichen Bildern. Da diese Informationen auf dem Schwarzmarkt oft zu nicht zu vernachlässigenden Preisen gehandelt werden, hat sich hier ein neuer Markt entwickelt. Black-Hat-Hacker oder Gruppen versuchen über Phishing-Mails oder Virenkampagnen so viele Daten wie Möglich zu sammeln, um diese im Anschluss zu verkaufen. Diese Tätigkeiten werden oft unter dem Schlagwort Cybercrime zusammengefasst. Umso kritischer in Hinsicht auf den möglichen finanziellen Schaden ist Cybercrime jedoch, wenn es um Unternehmen geht. Würde als Beispiel die Webseite eines Versandunternehmens von Hackern offline genommen werden, entstehen reale Verluste, da die Käufer auf andere Shops wechseln. Ähnlich kritisch wäre es, wenn streng vertrauliche Dokumente von Unternehmen, wie z.B. Konstruktionsskizzen oder Quellcode, durch Hacker erbeutet und an ein Konkurrenzunternehmen verkauft würden. So schätzt die Firma McAffee im Jahr 2014 den Verlust für die Wirtschaft durch "Cybercrime" auf bis zu 575 Milliarden USD\footnote{Siehe: $http://csis.org/files/attachments/140609\_McAfee\_PDF.pdf$}. 
Um diesem Trend entgegen zu wirken, müssen Unternehmen Maßnahmen ergreifen, welche das Schutzniveau erhöhen. Oft wird hier nur auf die technische Seite geachtet, z.B. auf das schnelle Patching von Servern. Dies ist sicherlich ein essentielle Bestandteil, oft wird jedoch die menschliche Komponente vernachlässigt, obwohl Social Engineering wohl zu den häufigsten Angriff Szenarien zählt. Dabei ist dies oft der erste Einfallspunkt für Cyberkriminelle. Für professionelle Hacker ist es kein Problem, eine E-Mail mit gefälschtem Absender und einem noch nicht erkannten Virus im Anhang zu verschicken. Öffnet der Mitarbeiter der Firma, welcher eine solche E-Mail empfängt, den Anhang ergeben sich gute Chancen für den Hacker Zugriff in das interne Netzwerk der Firma zu erlangen. Um dieses Risiko zu minimieren werden bei der Allianz Deutschland AG, im weiteren Verlauf AZD genannt, Awareness-Kampagnen mit Postern und Hinweistafeln, jedoch auch Schulungen mit Themen wie Spam und Phishing, organisiert. Diese Vorträge werden von wechselnden Mitarbeitern der Sicherheitsabteilung gehalten, welche im vorhinein Folien mit PowerPoint entwerfen und Demonstration entwickeln, an welchen das Themengebiet den Zuhörern praxisnah erläutern werden kann. Findet dieses ohne ein zentrales Programm statt, ergeben sich Problemfelder wie der Austausch, die Aktualität oder der Qualität der Vortragsmaterialien. Bei diesen und anderen Problemen soll das Security Awareness Framework, kurz SAF, Abhilfe schaffen.
\\
\\
\\
$[TODO ->]$Diese sind dann oft veraltet oder instabil. Zudem werden die Demos oft unter den Kollegen der Sicherheitsabteilungen nicht ausgetauscht. Dies hat mehrere Gründe, angefangen davon, dass diese oft nur für den Ersteller verständlich sind bis hin zu technischen Problemen wie der Größe oder dem Format der Virtuellen Maschine, auf welchen die Demos oft gespeichert sind. Diese Probleme soll durch das Security Awareness Framework gelöst werden.