\chapter{Szenario}
Bei einer Prüfung interner Regularien bei der Informationssicherheit wurde das Thema USB-Geräte in Verbindung mit Thin oder auch sogenannten Zero-Clients aufgegriffen. Hier soll aus gegebenen fachlichen Anlässen eine Möglichkeit geschaffen werden, lokale USB-Geräte, wie z.B. USB-Sticks oder USB-CD-Laufwerke an die virtuelle Maschine des Benutzers weiter zu leiten. In dieser Arbeit galt es, das Risiko zu prüfen und entsprechende Gegenmaßnahmen zu entwickeln.
Würde man das Durchstellen von USB-Geräten im Allgemeinen erlauben, so würden sich erhebliche Gefahren ergeben, welche unter \ref{GefBeiUSB} erläutert werden. Um dem vorzubeugen, soll auf Basis einer Policy, welche in \ref{Policies} weiter erläutert wird, das Durchstellen auf bestimmte Geräte begrenzt werden. Dies geschieht bei dem hier getesteten Produkt über die Filterung nach USB-Deskriptoren wie \textit{idVendor} und \textit{idProduct}, welche ein USB-Device, z.B. einen bestimmten USB-Stick oder ein USB-CD-Laufwerk, eindeutig identifizieren sollen. Diese Felder werden unter \ref{Deskriptoren} weiter erläutert.
Bei der durchgeführten Sicherheitsprüfung stellte sich jedoch heraus, dass USB-Deskriptoren keinen Sicherungen unterliegen und somit mit Hilfe bestimmter Geräte gezielt emuliert werden können. Diese Umgehung der Policy soll in diesem Dokument erläutert und aufgezeigt werden. Den Proof-Of-Concept finden Sie unter \ref{PoC}.

\begin{figure}[htbp]
\begin{tikzpicture}[
	scale=1,
	line width=0.25mm,
	every node/.style={
		scale=1, 
		text=THIblue},
	align=center,
	node distance=4cm,
	comp/.style={
		fill=white,
		rectangle,
		draw,
		minimum size=2.5cm},
	driver/.style={
		fill=white,
		rectangle,
		draw,
		yshift=2cm,
		xshift=-1cm},
	device/.style={
		fill=white,
		rectangle,
		draw},
	policie/.style={
		fill=white,
		rectangle,
		draw,
		yshift=-2cm,
		xshift=-1.5cm}
	]
	
	\node[comp] (thinclient) at (0,0){Thinclient};
	\node[driver] (thinclientUSB) [below right of=thinclient] {USB-Treiber};
	\node[device] (USBDevice) [right of=thinclientUSB, xshift=-1cm] {USB-Device};

	\node[comp] (hypervisor) [above of=thinclient] {Hypervisor};
	\node[driver] (hypervisorUSB) [below right of=hypervisor] {USB-Treiber};
	\node[policie] (hypervisorPol) [above right of=hypervisor] {Policie};
	
	\node[comp] (VM) [above of=hypervisor] {VM};
	\node[driver] (VMUSB) [below right of=VM] {USB-Treiber};
	\node[policie] (VMPol) [above right of=VM] {Policie};
		
	\draw[->]
		(USBDevice) --
			node[above]{1.}
				(thinclientUSB);	
	
	\draw[<->]
		(thinclientUSB) -- 
			node[right]{2.}
				++(0,-1) -| (USBDevice);
	
	\draw[->]
		(thinclient.125) -- 
			node[left]{3.}
				(hypervisor.235);
	
	\draw[<->]
		(hypervisor.10) -- 
			node[above]{4.}
				++(2.5,0) |- (hypervisorPol);
	
	\draw[->]
		(hypervisor.270) -- 
			node[left]{5.}
				(thinclient.90);
				
	\draw[->]
		(thinclient.55) -- 
			node[left]{6.}
				++(0,1.25)-|(hypervisorUSB);
	
	\draw[->]
		(hypervisor) --
			node[left]{7.}
				++(0,2.5)-|(VMUSB);
	
	\draw[<->]
		(VM.10) --
			node[above]{8.}
				++(2.5,0) |- (VMPol);
	
\end{tikzpicture}
\caption{Ablaufübersicht}
\label{fig:Ablauf}
\end{figure}

\begin{description}
	\item[Schritt 1: ] Ein USB-Gerät wird angesteckt.
	\item[Schritt 2: ] Der USB-Treiber des Thinclients bindet das Gerät ein.
	\item[Schritt 3: ] Der Thinclient leitet entsprechende Deskriptor-Felder oder das gesamte USB-Gerät an den Hypervisor weiter. 
	\item[Schritt 4: ] Der Hypervisor prüft die Deskriptor-Felder gegen die Hypervisor-Policy. Diese erlaubt entweder das Durchstellen oder verbietet es. Wird die Durchstellung verboten, wird der USB-Stick nicht an die Hypervisor-Umgebung weitergeleitet und damit würde der Ablauf enden. Im Folgenden wird angenommen, dass der USB-Stick weitergeleitet wird.
	\item[Schritt 5\&6: ] Der Hypervisor fordert den USB-Stick beim Thinclient an und bindet diesen ein.
	\item[Schritt 7: ] Der Hypervisor gibt das USB-Device an die VM weiter.
	\item[Schritt 8: ] Die VM prüft das USB-Device anhand der Deskriptor-Felder gegen die VM-Policy und bindet diesen entweder ein oder lehnt diesen ab.
\end{description}