\chapter{Universal Serial Bus}
Universal Serial Bus (USB) ist eine Schnittstelle, die so gut wie alle modernen Rechner besitzen. Es ist unter anderem möglich, darüber Geräte wie Kopfhörer und Joysticks aber auch Wechseldatenträger mit dem Rechner zu verbinden. Im letzteren Bereich ersetzt USB zunehmend die bisher vorherrschende CD/DVD-Technologie, da auf einem USB-Stick mehr Daten in höherer Geschwindigkeit gespeichert werden können und zudem die Geräte handlicher sind. Jedoch birgt die USB-Technologie auch Gefahren, die in den folgenden Sektionen beschrieben werden. Anschließend werden in \ref{Deskriptoren} die Deskriptoren beschrieben, die ein USB-Gerät mit sich bringt und die für die Prüfung gegen die Richtlinie eine besondere Rolle spielen.

\section{Gefahren bei USB}\label{GefBeiUSB}
USB-Geräte stellen Gefahren auf verschiedenen Ebenen dar. Zum einen werden USB-Sticks, zumindest in dem Szenario, das hier betrachtet wurde, von Dritten an Mitarbeiter gegeben. Das bedeutet, dass ein Dritter, insofern er die nötige kriminelle Energie aufweist, ein präpariertes Gerät einschicken könnte. Erschwerend kommt hinzu, dass der Mitarbeiter keine Möglichkeit hat, ein schädliches USB-Gerät von einem normalen zu unterscheiden. Im folgenden werden einige der Risiken dargestellt.

\subsection{Schadsoftware}
Schadsoftware sind eine wachsende Bedrohung. Vor einigen Jahren waren nur einige wenige Virenfamilien weit verbreitet. So konnten Virenschutzhersteller über signaturbasierte Suchalgorithmen nach bekannten Mustern suchen und Viren zuverlässig identifizieren. In den letzten Jahren zeichnet sich jedoch der Trend ab, dass Viren sich schneller weiterentwickelt werden und zudem oft polymorph programmiert sind, also ihr Aussehen bei einer Infektion verändern. Dadurch werden signaturbasierte Erkennungen immer ineffektiver und die Gefahr, dass ein Rechner unerkannt infiziert wird, steigt. Eine Infektion passiert zumeist über sogenannte Browserexploits, also präparierte Webseiten, die Lücken in der Software des Users nutzen, oder Anhänge an Mails, die Schadcode enthalten. In dem hier betrachteten Szenario schickt ein krimineller oder aber auch ein unwissender Dritter, dessen Rechner im Vorfeld von einem Virus infiziert wurde, einen USB-Stick mit einem Schadsoftware ein. Ein Beispiel für einen solchen Virus wäre ein Trojaner. Diese tarnen sich als normale Software, beinhalten aber auch Schadcodefunktionalität. \cite{Stamp2006} Will ein Benutzer das vermeintlich sinnvolle Programm installieren, wird im Hintergrund unbemerkt die Schadroutine mitinstalliert und gestartet. Dieser Schadcode hat oftmals Funktionalitäten wie Keylogger, Backdoors oder ein Rootkit. Als Beispiel könnte man hier den in \textit{Metasploit}\cite{Metasploit} enthaltenen \textit{Meterpreter} nennen, dessen Funktionen jedoch weit über die oben genannten hinaus gehen \cite{Meterpreter}. Hier muss also ein Benutzer einen USB-Stick einstecken und ein darauf befindliches Programm starten, damit sich der Virus installieren kann. Ist er bereits bekannt, könnte ein auf dem System installierter Virenscanner diesen erkennen und bestenfalls blockieren.
			
\subsection{Datenabfluss}
Neben den Gefahren von außen müssen auch sogenannte \textit{Inside-Threaths} beachtet werden. Dies wären Mitarbeiter, die z.B. interne IT-Systeme manipulieren, um sich Vorteile materieller oder immaterieller Art zu verschaffen. Bezogen auf USB wäre ein Risiko der Abfluss von vertraulichen oder wertvollen Daten, wenn also ein Mitarbeiter diese auf einem USB-Gerät speichert und aus dem Einflussbereich des Unternehmens bringt. Anschließend könnte er sich an diesen bereichern, falls es sich zum Beispiel um Bankdaten handelt. Andere denkbare Beispiele sind Kundendaten, Benutzerkonten, Geschäftsberichte vor der Veröffentlichung oder sonstige Unternehmensgeheimnisse. Diese können oft für Geld in einschlägigen Bereichen des Internets verkauft oder bei Geschäftsberichten zur Manipulation am Finanzmarkt genutzt werden. Auch wäre eine Abwerbung eines Mitarbeiters von einem anderen Unternehmen für Industriespionage denkbar.
Eine weitere Bedrohung sind Geheimdienste, die Personen in ein Unternehmen einschleusen oder Mitarbeiter abwerben, um Daten über die Kunden zu sammeln. Dies wurde erst kürzlich durch von Edward Snowden veröffentlichte Dokumente publik.\cite{Snowden2}

\subsection{Exploits auf Treiberebene}
Für den Mitarbeiter noch schwieriger zu entdecken sind Exploits auf Treiberebene. Dieses Vorgehen ist relativ neu. Es werden dabei Lücken im Treiber des Geräts ausgenutzt, um Schadcode auszuführen. Hierzu muss ein Benutzer einen manipulierten USB-Stick nur einstecken. Es bedarf im Gegensatz zu normalen Viren keiner weiteren Interaktion des Users, da der Computer automatisch mit dem USB-Device kommuniziert, um die Funktionen des Geräts zu erfahren und eventuell benötigte Treiber zu installieren. Hier beginnt das Gerät jedoch bereits, bestimmte schädliche Zeichenfolgen an den  Computer zu senden, die vom Treiber interpretiert werden und unter Umständen einen Buffer Overflow  oder eine andere Schwachstelle ausnutzen können\cite{BadUSB}. Durch diese Lücken kann dann auf dem Rechner des Benutzers unbemerkt Schadcode ausgeführt werden.

%\section{Technische Ablauf beim Einstecken eines USB-Devices}
%Dieser kurze Abschnitt gibt eine genauere, jedoch immer noch oberflächlich gehaltene Beschreibung der Schritte 2., 6. und 7. der Grafik \ref{fig:Ablauf}. Wird ein USB-Gerät eingesteckt, bekommt dieser über den USB-Anschluss Strom und sendet ein Ankündigungspaket. Der Computer reagiert darauf und fordert Informationen, also die in \ref{Deskriptoren} beschriebenen Felder, an. Das USB-Gerät überträgt diese an den PC. \cite{USBDesk}

\newpage
\section{Deskriptoren}\label{Deskriptoren}
Die USB-Spezifikation, die von dem USB Implementers Forum, Inc.\cite{USBAbout} festgelegt wird, sieht folgende Felder vor, die Informationen zu dem Gerät beinhalten.
\begin{wrapfigure}{l}{0pt}
\begin{tikzpicture}[scale=1, text=THIblue]
	\draw (0,0) rectangle (0.5,0.5);
	\draw (0.25, 0.25) node {1};
	\draw (0.5, 0.25) node[right]{bLength};
	
	\draw (0,0.5) rectangle (0.5,0.5);
	\draw (0.25, 0.75) node {1};
	\draw (0.5, 0.75) node[right]{bDescriptorType};
	
	\draw (0,1) rectangle (0.5,1);
	\draw (0.25,1.5) node {2};
	\draw (0.5,1.5) node[right]{bcdUSB};
	
	\draw (0,2) rectangle (0.5,0.5);
	\draw (0.25,2.25) node {1};
	\draw (0.5,2.25) node[right]{bDeviceClass};

	\draw (0,2.5) rectangle (0.5,0.5);
	\draw (0.25,2.75) node {1};
	\draw (0.5,2.75) node[right]{bDeviceSubClass};

	\draw (0,3) rectangle (0.5,0.5);
	\draw (0.25,3.25) node {1};
	\draw (0.5,3.25) node[right]{bDeviceProtocol};

	\draw (0,3.5) rectangle (0.5,0.5);
	\draw (0.25,3.75) node {1};
	\draw (0.5,3.75) node[right]{bMaxPacketSize};
	
	\draw (0,4) rectangle (0.5,1);
	\draw (0.25,4.5) node {2};
	\draw (0.5,4.5) node[right]{idVendor};
	
	\draw (0,5) rectangle (0.5,1);
	\draw (0.25,5.5) node {2};
	\draw (0.5,5.5) node[right]{idProduct};
	
	\draw (0,6) rectangle (0.5,1);
	\draw (0.25,6.5) node {2};
	\draw (0.5,6.5) node[right]{bcdDevice};
	
	\draw (0,7) rectangle (0.5,0.5);
	\draw (0.25,7.25) node {1};
	\draw (0.5,7.25) node[right]{iManufacturer};
	
	\draw (0,7.5) rectangle (0.5,0.5);
	\draw (0.25,7.75) node {1};
	\draw (0.5,7.75) node[right]{iProduct};
	
	\draw (0,8) rectangle (0.5,0.5);
	\draw (0.25,8.25) node {1};
	\draw (0.5,8.25) node[right]{iSerialNumber};
	
	
	\draw (0,8.5) rectangle (0.5,0.5);
	\draw (0.25,8.75) node {1};
	\draw (0.5,8.75) node[right]{bNumConfigurations};
	
	\draw (0,9) rectangle (0.5,0.5);
	
	\draw (0.25, 9) node[rotate=90, right] {Bytes};
\end{tikzpicture}
\label{fig:usbDeskriptoren}
\end{wrapfigure}

Dies umfasst die technischen Informationen, wie die Länge der gesamten Felder im \textit{bLength} oder das Protokoll des Geräts im \textit{bDeviceProtocol} über Informationen für das Betriebssystem wie \textit{idVendor}, \textit{idProduct}, \textit{bDeviceSubClass} und \textit{bDeviceClass}. Diese Felder mit der jeweiligen Länge sind in der Grafik dargestellt. Ein Feld hat dabei zwischen ein und zwei Bytes. Die Felder \textit{bDeviceClass}, \textit{bDeviceSubClass}, \textit{bDeviceProtocol} sowie \textit{idVendor} werden vom Hersteller befüllt.\cite{USBDesk} Das Betriebssystem nutzt die Felder meist, um Treiber zu suchen oder auch das angeschlossene USB-Gerät gegen die Policy-Einstellungen zu prüfen. Um eigene Werte bei \textit{idProduct} oder \textit{idVendor}-Felder zu nutzen und damit sicher gestellt ist, dass nicht mehrere Hersteller dieselbe \textit{idProduct} verwenden, müssen die Adressbereiche der \textit{idProduct} bei dem USB Implementers Forum, Inc. gekauft werden. Dazu gibt es zwei Möglichkeiten. Man kann entweder ein Mitglied des Forums werden oder für einen einmaligen Betrag einen Adressraum erstehen. Im zweiten Fall darf man jedoch nicht das offizielle USB-Logo verwenden. \cite{USBVendor} Im folgenden werden die für dieses Dokument interessanten Felder weiter erläutert:

%\begin{figure}[h]
%	\setlength{\unitlength}{0.14in} % selecting unit length
%	\centering % used for centering Figure
%	\begin{picture}(36,10) % picture environment with the size (dimensions)
%% 32 length units wide, and 15 units high.
%		\put(0,4){BlaBLab}
%		\put(0,0){\framebox(2,3){1}}
%		\put(2,0){\framebox(2,3){1}}
%		\put(4,0){\framebox(4,3){1}}
%		\put(8,0){\framebox(2,3){1}}
%		\put(10,0){\framebox(2,3){1}}
%		\put(12,0){\framebox(2,3){1}}
%		\put(14,0){\framebox(2,3){1}}
%		\put(16,0){\framebox(4,3){1}}
%		\put(20,0){\framebox(4,3){1}}
%		\put(24,0){\framebox(4,3){1}}
%		\put(28,0){\framebox(2,3){1}}
%		\put(30,0){\framebox(2,3){1}}
%		\put(32,0){\framebox(2,3){1}}
%		\put(34,0){\framebox(2,3){1}}
%		\put(23,4){\framebox(6,3){$H_{C}(q)$}}
%		\put(0,5.5){\vector(1,0){3}}
%		\put(19.5,6.5) {$x_{C}(k)$}
%	\end{picture}
%	\caption{Aufbau der USB-Descriptoren}
%	% title of the Figure
%	\label{fig:lnlblock}
%	% label to refer figure in text
%\end{figure}			
$ $\\\\\\
\begin{description}
	\item[idVendor: ] Der Inhalt des \textit{idVendor}-Feld wird von der USB Implementers Forum, Inc. festgelegt. Das Feld ist 2 Byte lang und jeder Wert ist genau einem Hersteller zugeordnet. Ersteht ein Unternehmen einen \textit{idVendor}-Wert, kann er, solange er diesen \textit{idVendor}-Wert nutzt, frei über das \textit{idProduct}-Feld verfügen.
	
	\item[idProduct: ] Das \textit{idProduct}-Feld wird von dem Unternehmen vergeben, das einen Wert im \textit{idVendor}-Feld gekauft hatte. Es ist ebenfalls 2 Bytes lang. Damit kann ein Unternehmen bis zu $2^{16}$ verschiedene Produkte beschreiben.
	
	\item[bInterfaceClass \& bInterfaceSubClass \& bInterfaceProtocol: ] Diese Felder sind in der oberen Aufstellung nicht enthalten, da sie nicht über das Device, sondern über das \textit{USB-Interface} Auskunft geben. Sie beschreiben die Klasse, die Unterklasse und das Protokoll, über das mit dem Interface kommuniziert werden soll.
	
\end{description}