\chapter{Fazit und Gegenmaßnahmen}
Da der USB-Standard keine Möglichkeit bietet Geräte fehlerfrei, zum Beispiel über Signaturen, zu identifizieren kann man sich nicht auf diese Ebene verlassen und muss man die Gefahren direkt eindämmen. Jedoch ist auch dies im Bezug auf die Exploits auf Treiberebene sehr schwer. Die einfachste und sicherste Methode wäre, die Benutzung von USB-Ports in einem Unternehmen per Richtlinie zu verbieten und die Ports eventuell sogar noch physikalisch zu Versiegeln. Hier hätte man natürlich den Nachteil, dass USB-Geräte nicht mehr direkt genutzt werden könnten. Als Lösung für dieses Problem, könnte man eine Art Schleusen-System für USB-Geräte aufgebaut werden. So könnte wenn ein USB-Stick an die Firma geschickt wird, dieser in dem Terminal-Server eingebunden und die Daten an den gewünschten Empfänger weiterreicht werden. Die Vorteile sind hier, dass falls Viren auf dem USB-Stick enthalten sind, diese vorher am Terminalserver sowie bei der Netzwerkübertragung von mehreren verschiedenen Virenscanner überprüft sowie heuristischen Analysten unterworfen werden könnten. Ebenso würde bei einem manipulierten USB-Stick nicht der Rechner des Mitarbeiters sondern nur der Schleusen-PC infiziert. Trifft man hier entsprechende Sicherheitsmaßnahmen wie die Abschottung vom Internet und einen regelmäßigen Tausch sowie eine regelmäßige Neuinstallation des Schleusenrechners, kann man das Risiko hier relativ gering halten. Der Mitarbeiter würde in diesem Fall nur die Dateien bekommen und wäre von der Manipulation auf Treiberebene nicht betroffen. Zudem sind die im Vorfeld getroffenen Sicherheitsmaßnahmen für den Mitarbeiter transparent.